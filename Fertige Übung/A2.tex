\documentclass[11pt,a4paper]{article}
\usepackage[utf8]{inputenc}
\usepackage{german}
\date{\today}
\author{Sebastian Buchmann}
\title{ \LaTeX ~ Übung 5}

\begin{document}
\maketitle
\section{1 Abschnitt}

Mein erster Abschnitt

\section{Ich war hier}
Hallo Yanik Weber war hier!


\section{Tabelle}
Hier kommt die Tabelle \ref{tab:Punkte}
\begin{table} [h]
\centering
\begin{tabular} {c|c|c|c} 

{} & {Punkte erhalten}  & {Punkte möglich} & {\%}   \\
\hline

Aufgabe 1 & 4 & 4 & 1 \\
Aufgabe 2 & 3 & 3 & 1 \\
Aufgabe 3 & 3 & 3 & 1 \\


\end{tabular}

\caption{Hier ist die beschriftung}
\label{tab:Punkte}
\end{table}
\section{Formeln}

\subsection{Pytagoras}
 Der Satz des Pytagoras lautet $ a^2 + b^2 = c^2$ Für die Hypothenuse gilt folglich $c = \sqrt{a^2+b^2}$
 
\subsection{Summen}

Formel für die Summe: \\
\begin{equation}
 s= \sum\limits_{i=1}^{n} = \frac{n*(n+1)}{2}   
\centering
\end{equation}



\end{document}
